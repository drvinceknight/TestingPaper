Research software has fundamentally different life cycles from commercial software. While this is (implcitly) recognised by authors and funders, its implications for the testing regime have not been clearly articulated. Here the authors from several UK research institutions have pooled their views on the testing strategies appropriate to research software at various stages of its evolution. What is sufficient for a program being used by one research student to underpin a thesis is probably insufficient for a program being used by many people, most of whom never read the source, in many institutions, on a wide range of computers.
